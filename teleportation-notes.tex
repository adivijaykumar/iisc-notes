\documentclass[twoside]{article}
\setlength{\oddsidemargin}{0.00 in}
\setlength{\evensidemargin}{0.00 in}
\setlength{\topmargin}{-0.6 in}
\setlength{\textwidth}{6.5 in}
\setlength{\textheight}{8.5 in}
\setlength{\headsep}{0.75 in}
\setlength{\parindent}{0 in}
\setlength{\parskip}{0.1 in}

%
% ADD PACKAGES here:
%

\usepackage{amsmath,amsfonts,graphicx}
\usepackage{amsthm}
\usepackage{amstext}
\usepackage{amssymb}
\usepackage{float}

\newcommand{\sgn}{{\rm sign}}
\newcommand{\ret}{\nonumber \\}
%
% The following commands set up the lecnum (lecture number)
% counter and make various numbering schemes work relative
% to the lecture number.
%
\newcounter{lecnum}
\renewcommand{\thepage}{\thelecnum-\arabic{page}}
\renewcommand{\thesection}{\thelecnum.\arabic{section}}
\renewcommand{\theequation}{\thelecnum.\arabic{equation}}
\renewcommand{\thefigure}{\thelecnum.\arabic{figure}}
\renewcommand{\thetable}{\thelecnum.\arabic{table}}

\def\CA{{\cal{A}}}
\def\CB{{\cal{B}}}
\def\CC{{\cal{C}}}
\def\CD{{\cal{D}}}
\def\CE{{\cal{E}}}
\def\CF{{\cal{F}}}
\def\CG{{\cal{G}}}
\def\CH{{\cal{H}}}
\def\CI{{\cal{I}}}
\def\CJ{{\cal{J}}}
\def\CK{{\cal{K}}}
\def\CL{{\cal{L}}}
\def\CM{{\cal{M}}}
\def\CN{{\cal{N}}}
\def\CO{{\cal{O}}}
\def\CP{{\cal{P}}}
\def\CQ{{\cal{Q}}}
\def\CR{{\cal{R}}}
\def\CS{{\cal{S}}}
\def\CT{{\cal{T}}}
\def\CU{{\cal{U}}}
\def\CV{{\cal{V}}}
\def\CW{{\cal{W}}}
\def\CX{{\cal{X}}}
\def\CY{{\cal{Y}}}
\def\CZ{{\cal{Z}}}
\def\B{{\cal{B}}}
\def\CA{{\cal{A}}}
\def\t{\tau}
\def\cg{complexity geometry}
\def\as{analog system}
\def\kl{$k$-local}
\def\sa{$\CA$}
\def\qt{quantum teleportation \ }

\DeclareMathOperator{\Tr}{Tr}
\def\eq{&=&}
\def\d{\partial}
\def\dt{\partial_{\tau}}
\def\ds{\partial_{\sigma}}
\def\la{\langle}
\def\ra{\rangle}
\def\lb{\label}
\def\simleq{\; \raise0.3ex\hbox{$<$\kern-0.75em
\raise-1.1ex\hbox{$\sim$}}\; }
\def\simgeq{\; \raise0.3ex\hbox{$>$\kern-0.75em
\raise-1.1ex\hbox{$\sim$}}\; }
\def\s{$\sigma$}
\renewcommand{\thefootnote}{\fnsymbol{footnote}}
\renewcommand{\thanks}[1]{\footnote{#1}} % Use this for footnotes
\newcommand{\starttext}{
\setcounter{footnote}{0}
\renewcommand{\thefootnote}{\arabic{footnote}}}
\renewcommand{\theequation}{\thesection.\arabic{equation}}
\newcommand{\be}{\begin{equation}}
\newcommand{\bea}{\begin{eqnarray}}
\newcommand{\eea}{\end{eqnarray}}
\newcommand{\beq}{\begin{equation}}
\newcommand{\ee}{\end{equation}}
%
% The following macro is used to generate the header.
%
\newcommand{\lecture}[4]{
   \pagestyle{myheadings}
   \thispagestyle{plain}
   \newpage
   \setcounter{lecnum}{#1}
   \setcounter{page}{1}
   \noindent
   \begin{center}
   \framebox{
      \vbox{\vspace{2mm}
    \hbox to 6.28in { {\bf Thesis Notes - Aditya Vijaykumar
    \hfill Semester I 2017-18} }
       \vspace{4mm}
       \hbox to 6.28in { {\Large \hfill #1: #2  \hfill} }
       \vspace{2mm}
       
      }
   }
   \end{center}
   \markboth{Lecture #1: #2}{Lecture #1: #2}

   \vspace*{4mm}
}
%
% Convention for citations is authors' initials followed by the year.
% For example, to cite \bf a paper by Leighton and Maggs you would type
% \cite{LM89}, and to cite \bf a paper by Strassen you would type \cite{S69}.
% (To avoid bibliography problems, for now we redefine the \cite command.)
% Also commands that create \bf a suitable format for the reference list.
\renewcommand{\cite}[1]{[#1]}
\def\beginrefs{\begin{list}%
        {[\arabic{equation}]}{\usecounter{equation}
         \setlength{\leftmargin}{2.0truecm}\setlength{\labelsep}{0.4truecm}%
         \setlength{\labelwidth}{1.6truecm}}}
\def\endrefs{\end{list}}
\def\bibentry#1{\item[\hbox{[#1]}]}
\def\mean#1{\left< #1 \right>}

%Use this command for \bf a figure; it puts \bf a figure in wherever you want it.
%usage: \fig{NUMBER}{SPACE-IN-INCHES}{CAPTION}
\newcommand{\fig}[3]{
      \vspace{#2}
      \begin{center}
      Figure \thelecnum.#1:~#3
      \end{center}
  }
% Use these for theorems, lemmas, proofs, etc.
\newtheorem{theorem}{Theorem}[lecnum]
\newtheorem{lemma}[theorem]{Lemma}
\newtheorem{proposition}[theorem]{Proposition}
\newtheorem{claim}[theorem]{Claim}
\newtheorem{corollary}[theorem]{Corollary}
\newtheorem{definition}[theorem]{Definition}
%\newenvironment{proof}{{\bf Proof:}}{\hfill\rule{2mm}{2mm}}

% **** IF YOU WANT TO DEFINE ADDITIONAL MACROS FOR YOURSELF, PUT THEM HERE:
\newcommand\E{\mathbb{E}}


\begin{document}
%FILL IN THE RIGHT INFO.
%\lecture{**LECTURE-NUMBER**}{**TOPIC**}{**LECTURER**}{**LITE**}
\lecture{1}{Teleportation}{Aditya Vijaykumar}{scribe-name}
%\footnotetext{These notes are partially based on those of Nigel Mansell.}

% **** YOUR NOTES GO HERE:
\section{Classical Message Sending}
How does one send a message classically? Let's suppose Alice and Bob have a bit each, and are far away from each other. Each bit can take value 0 or 1, but lets say that both bits have the same value. Alice and Bob could have met at some time in the past, created their bits, assigned them the same value and then moved on to their respective present positions.

Alice has one more bit with her - and she wishes to send the state of this bit to Bob. Lets call the bit the \textit{teleportee} bit. She proceeds to compare both bits in her possession. If both the bits in her possession are of the same value she sends Bob a classical message reading \textit{same}; if the bits have different value the message reads \textit{different}. Bob, on looking at the message, flips the value of the bit if the message reads \textit{different} and does not if it reads \textit{same}. So, whatever the initial value of Bob's bit, he always ends up with his bit in the same state as that of the teleportee bit.

Now lets suppose Eve intercepts the message sent from Alice to Bob. Obviously, she cannot know anything about the original state of the teleportee solely from this information. But, in principle, she can probe the brain waves or the gravitational waves around Alice and in the environment to know exactly what the state of the bit was. So, in a sense, we could say that the information did in fact exist in the environment between Alice and Bob and we can measure it in principle.

\section{Quantum Teleportation}
Lets consider the quantum analog of the above protocol. Instead of bits, we consider qubits. Lets say, without loss of generality, Alice and Bob begin in the normalized entangled state $$\frac{1}{\sqrt{2}}|00 \ra + \frac{1}{\sqrt{2}}|11 \ra$$

Alice and Bob are currently far away from each other. Alice has a \textit{teleportee} qubit in the unit normalized state $$|\phi \ra = \alpha |0 \ra + \beta |1 \ra$$

We can write the combined state of the three qubits as follows, with the first qubit denoting the teleportee, second denoting Alice's and third denoting Bob's. 
\begin{equation}
\frac{\alpha}{\sqrt{2}}|000 \ra + \frac{\alpha}{\sqrt{2}}|011 \ra + \frac{\beta}{\sqrt{2}}|100 \ra + \frac{\beta}{\sqrt{2}}|111 \ra \label{combstate}
\end{equation}

We shall now measure the two qubits in Alice's possession (ie. first and second qubits) in the Bell Basis. The normalized Bell basis vectors are given by

$$|B_1 \ra = \frac{1}{\sqrt{2}}|00 \ra + \frac{1}{\sqrt{2}}|11 \ra ; |B_2 \ra = \frac{1}{\sqrt{2}}|01 \ra + \frac{1}{\sqrt{2}}|10 \ra;
|B_3 \ra = \frac{1}{\sqrt{2}}|00 \ra - \frac{1}{\sqrt{2}}|11 \ra ; |B_4 \ra = \frac{1}{\sqrt{2}}|01 \ra - \frac{1}{\sqrt{2}}|10 \ra$$

We relate the computational basis to the Bell basis as
$$|00 \ra = \frac{|B_1 \ra + |B_3 \ra}{\sqrt{2}} ; |10 \ra = \frac{|B_2 \ra + |B_4 \ra}{\sqrt{2}} ; |01 \ra = \frac{|B_2\ra - |B_4\ra}{\sqrt{2}}; |11 \ra = \frac{|B_1\ra - |B_3\ra}{\sqrt{2}}$$

Substituting these expressions in \ref{combstate}, one can write the combined state as 
$$\frac{|B_1 \ra(\alpha|0 \ra + \beta |1 \ra) + |B_2 \ra (\alpha |1 \ra + \beta |0 \ra) + |B_3 \ra (\alpha|0 \ra - \beta |1 \ra) |B_4\ra (\alpha|1 \ra - \beta |0 \ra)}{2}$$
We can clearly see from this state that any measurement in the bell basis will give $|B_1 \ra, |B_2 \ra, |B_3 \ra$ or $|B_4 \ra$ with equal probability (25 percent). After the measurement, Bob's qubit is left in the state $\alpha|0 \ra + \beta |1 \ra$, $\alpha|1 \ra + \beta |0 \ra$, $\alpha|0 \ra - \beta |1 \ra$, and $\alpha|1 \ra + \beta |0 \ra$ respectively.

Alice can, through a classical message, send the result of her measurement to Bob. Bob can then, depending on the classical message, apply the appropriate unitary operator on his state to take it to $(\alpha|0 \ra + \beta|1 \ra)$, the original state of the teleportee qubit. Thus, the teleportation has been completed successfully.

A few comments about this protocol :-
\begin{itemize}
\item Lets say Eve intercepts the classical message sent by Alice. Obviously, she cannot know anything about the original state of the teleportee solely from this information. But now, even if she probes the environment, she cannot extract any information about the qubits. This is because the entangled state that we started out with is maximally entangled. By the monogamy principle of entanglement, these qubits can now not be entangled with anything else. 
\item This also implies that the teleportee qubit got teleported to Bob with no information about it existing anywhere in between Alice and Bob. This tenet of non-locality of information is crucial to quantum mechanics.
\item The process of measuring the qubits destroys the existing entanglement between the qubits.
\end{itemize}
\end{document}