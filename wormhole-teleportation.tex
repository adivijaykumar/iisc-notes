%
% This is the LaTeX template file for lecture notes for CS267,
% Applications of Parallel Computing.  When preparing 
% LaTeX notes for this class, please use this template.
%
% To familiarize yourself with this template, the body contains
% some examples of its use.  Look them over.  Then you can
% run LaTeX on this file.  After you have LaTeXed this file then
% you can look over the result either by printing it out with
% dvips or using xdvi.
%

\documentclass[twoside]{article}
\setlength{\oddsidemargin}{0.25 in}
\setlength{\evensidemargin}{-0.25 in}
\setlength{\topmargin}{-0.6 in}
\setlength{\textwidth}{6.5 in}
\setlength{\textheight}{8.5 in}
\setlength{\headsep}{0.75 in}
\setlength{\parindent}{0 in}
\setlength{\parskip}{0.1 in}

%
% ADD PACKAGES here:
%

\usepackage{amsmath,amsfonts,graphicx}
\usepackage{amsthm}
\usepackage{amstext}
\usepackage{amssymb}

\newcommand{\sgn}{{\rm sign}}
\newcommand{\ret}{\nonumber \\}
%
% The following commands set up the lecnum (lecture number)
% counter and make various numbering schemes work relative
% to the lecture number.
%
\newcounter{lecnum}
\renewcommand{\thepage}{\thelecnum-\arabic{page}}
\renewcommand{\thesection}{\thelecnum.\arabic{section}}
\renewcommand{\theequation}{\thelecnum.\arabic{equation}}
\renewcommand{\thefigure}{\thelecnum.\arabic{figure}}
\renewcommand{\thetable}{\thelecnum.\arabic{table}}

\DeclareMathOperator{\Tr}{Tr}
\def\eq{&=&}
\def\d{\partial}
\def\dt{\partial_{\tau}}
\def\ds{\partial_{\sigma}}
\def\la{\langle}
\def\ra{\rangle}
\def\lb{\label}
\def\simleq{\; \raise0.3ex\hbox{$<$\kern-0.75em
\raise-1.1ex\hbox{$\sim$}}\; }
\def\simgeq{\; \raise0.3ex\hbox{$>$\kern-0.75em
\raise-1.1ex\hbox{$\sim$}}\; }
\def\s{$\sigma$}
\renewcommand{\thefootnote}{\fnsymbol{footnote}}
\renewcommand{\thanks}[1]{\footnote{#1}} % Use this for footnotes
\newcommand{\starttext}{
\setcounter{footnote}{0}
\renewcommand{\thefootnote}{\arabic{footnote}}}
\renewcommand{\theequation}{\thesection.\arabic{equation}}
\newcommand{\be}{\begin{equation}}
\newcommand{\bea}{\begin{eqnarray}}
\newcommand{\eea}{\end{eqnarray}}
\newcommand{\beq}{\begin{equation}}
\newcommand{\ee}{\end{equation}}
%
% The following macro is used to generate the header.
%
\newcommand{\lecture}[4]{
   \pagestyle{myheadings}
   \thispagestyle{plain}
   \newpage
   \setcounter{lecnum}{#1}
   \setcounter{page}{1}
   \noindent
   \begin{center}
   \framebox{
      \vbox{\vspace{2mm}
    \hbox to 6.28in { {\bf Thesis Notes - Aditya Vijaykumar
		\hfill Semester I 2017-18} }
       \vspace{4mm}
       \hbox to 6.28in { {\Large \hfill #1: #2  \hfill} }
       \vspace{2mm}
       
      }
   }
   \end{center}
   \markboth{Lecture #1: #2}{Lecture #1: #2}

   \vspace*{4mm}
}
%
% Convention for citations is authors' initials followed by the year.
% For example, to cite \bf a paper by Leighton and Maggs you would type
% \cite{LM89}, and to cite \bf a paper by Strassen you would type \cite{S69}.
% (To avoid bibliography problems, for now we redefine the \cite command.)
% Also commands that create \bf a suitable format for the reference list.
\renewcommand{\cite}[1]{[#1]}
\def\beginrefs{\begin{list}%
        {[\arabic{equation}]}{\usecounter{equation}
         \setlength{\leftmargin}{2.0truecm}\setlength{\labelsep}{0.4truecm}%
         \setlength{\labelwidth}{1.6truecm}}}
\def\endrefs{\end{list}}
\def\bibentry#1{\item[\hbox{[#1]}]}

%Use this command for \bf a figure; it puts \bf a figure in wherever you want it.
%usage: \fig{NUMBER}{SPACE-IN-INCHES}{CAPTION}
\newcommand{\fig}[3]{
			\vspace{#2}
			\begin{center}
			Figure \thelecnum.#1:~#3
			\end{center}
	}
% Use these for theorems, lemmas, proofs, etc.
\newtheorem{theorem}{Theorem}[lecnum]
\newtheorem{lemma}[theorem]{Lemma}
\newtheorem{proposition}[theorem]{Proposition}
\newtheorem{claim}[theorem]{Claim}
\newtheorem{corollary}[theorem]{Corollary}
\newtheorem{definition}[theorem]{Definition}
%\newenvironment{proof}{{\bf Proof:}}{\hfill\rule{2mm}{2mm}}

% **** IF YOU WANT TO DEFINE ADDITIONAL MACROS FOR YOURSELF, PUT THEM HERE:
\newcommand{\be}{\begin{equation}}
\newcommand{\ee}{\end{equation}}
\newcommand\E{\mathbb{E}}

\begin{document}
%FILL IN THE RIGHT INFO.
%\lecture{**LECTURE-NUMBER**}{**TOPIC**}{**LECTURER**}{**LITE**}
\lecture{2}{Teleportation Through the Wormhole - Susskind, Zhao}{Aditya Vijaykumar}{scribe-name}
%\footnotetext{These notes are partially based on those of Nigel Mansell.}

% **** YOUR NOTES GO HERE:

% Some general latex examples and examples making use of the
% macros follow.  
%**** IN GENERAL, BE BRIEF. LONG SCRIBE NOTES, NO MATTER HOW WELL WRITTEN,
%**** ARE NEVER READ BY ANYBODY.

The characters in this $gedanken$ story are Alice, Bob, Charlie and Eve.

Charlie produces two identical bits of information, with the rule that both the bits are either 0 or 1. Let's call these bits $\bf A$ and $\bf B$ respectively.

Charlie hands over $\bf A$ and $\bf B$ to Alice and Bob respectively. He then asks them to go far apart, and they obey readily.

Now Alice has another bit $\bf T$ in addition to $\bf A$, which she wishes to send to Bob. So she sends a message through some communication channel. If the message says \textit{same}, Bob retains the state of $\bf B$ and if it says \textit{different}, he flips the state. Hence, Bob always ends up with $\bf B$ in the configuration same as that of $\bf T$.

But Eve wants to know what $\bf T$ is. She tries to intercept the communication. But since the message only says \textit{same} or \textit{different}, she cannot have any information about the state of $\bf T$. But she has ways and means. She can try to coax Charlie, and extract from his memory the configuration of the bits he had created. Even if Charlie's memory had been erased, the information in his brain has been emitted into the environment (maybe as gravitational radiation \ldots, you get the gist). Hence, Eve could \textit{in principle} detect that information.

So much for classical information.

$$|0\ra_A |0\ra_B + |1\ra_A|1\ra_B$$ This in the entangled state of qubits that Charlie produced. Again, he hands over $\bf A$ and $\bf B$ to Alice and Bob respectively. Alice has $\bf T$ which has the state $$|\Phi\ra_T \equiv \Phi(0)|0\ra_T + \Phi(1) |1\ra_T$$.

The two qubits that Alice has is a linear combination any one of the following (Bell Basis) :-
\bea
|1\ra \eq |00\ra + |11\ra   \cr  
|x\ra \eq |10\ra + |01\rangle \cr 
|y\ra \eq |10\ra - |01\ra   \cr  
|z\ra \eq |00\ra - |11\ra   
\eea
Measurement gives Alice one of the outcomes $(1,x,y,z)$. She writes this up and sends it to Bob, who applies the corresponding operator on his qubit. Hence, his qubit is always $$|\Phi\ra_B \equiv \Phi(0)|0\ra_B + \Phi(1) |1\ra_B$$ ie. the same as $\bf T$.

But, can Eve intercept the message and find out about $\bf T$? No! Bell's basis states are maximally entangled. Entanglement is monogamous, in the sense that if $\bf A$ and $\bf B$ are maximally entangled. they \textit{cannot} be correlated with any other factors - Charlie's brain, gravitational radiation, or anything else.

But if one believes ER = EPR, then these maximally entangled EPR pairs should be connected by a microscopic Einstein-Rosen bridge.

\end{document}