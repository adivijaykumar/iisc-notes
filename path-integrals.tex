\documentclass[twoside]{article}
\setlength{\oddsidemargin}{0.00 in}
\setlength{\evensidemargin}{0.00 in}
\setlength{\topmargin}{-0.6 in}
\setlength{\textwidth}{6.5 in}
\setlength{\textheight}{8.5 in}
\setlength{\headsep}{0.75 in}
\setlength{\parindent}{0 in}
\setlength{\parskip}{0.1 in}

%
% ADD PACKAGES here:
%

\usepackage{amsmath,amsfonts,graphicx}
\usepackage{amsthm}
\usepackage{amstext}
\usepackage{amssymb}
\usepackage{float}

\newcommand{\sgn}{{\rm sign}}
\newcommand{\ret}{\nonumber \\}
%
% The following commands set up the lecnum (lecture number)
% counter and make various numbering schemes work relative
% to the lecture number.
%
\newcounter{lecnum}
\renewcommand{\thepage}{\thelecnum-\arabic{page}}
\renewcommand{\thesection}{\thelecnum.\arabic{section}}
\renewcommand{\theequation}{\thelecnum.\arabic{equation}}
\renewcommand{\thefigure}{\thelecnum.\arabic{figure}}
\renewcommand{\thetable}{\thelecnum.\arabic{table}}

\def\CA{{\cal{A}}}
\def\CB{{\cal{B}}}
\def\CC{{\cal{C}}}
\def\CD{{\cal{D}}}
\def\CE{{\cal{E}}}
\def\CF{{\cal{F}}}
\def\CG{{\cal{G}}}
\def\CH{{\cal{H}}}
\def\CI{{\cal{I}}}
\def\CJ{{\cal{J}}}
\def\CK{{\cal{K}}}
\def\CL{{\cal{L}}}
\def\CM{{\cal{M}}}
\def\CN{{\cal{N}}}
\def\CO{{\cal{O}}}
\def\CP{{\cal{P}}}
\def\CQ{{\cal{Q}}}
\def\CR{{\cal{R}}}
\def\CS{{\cal{S}}}
\def\CT{{\cal{T}}}
\def\CU{{\cal{U}}}
\def\CV{{\cal{V}}}
\def\CW{{\cal{W}}}
\def\CX{{\cal{X}}}
\def\CY{{\cal{Y}}}
\def\CZ{{\cal{Z}}}
\def\B{{\cal{B}}}
\def\CA{{\cal{A}}}
\def\t{\tau}
\def\cg{complexity geometry}
\def\as{analog system}
\def\kl{$k$-local}
\def\sa{$\CA$}
\def\qt{quantum teleportation \ }

\DeclareMathOperator{\Tr}{Tr}
\def\eq{&=&}
\def\d{\partial}
\def\dt{\partial_{\tau}}
\def\ds{\partial_{\sigma}}
\def\la{\langle}
\def\ra{\rangle}
\def\lb{\label}
\def\simleq{\; \raise0.3ex\hbox{$<$\kern-0.75em
\raise-1.1ex\hbox{$\sim$}}\; }
\def\simgeq{\; \raise0.3ex\hbox{$>$\kern-0.75em
\raise-1.1ex\hbox{$\sim$}}\; }
\def\s{$\sigma$}
\renewcommand{\thefootnote}{\fnsymbol{footnote}}
\renewcommand{\thanks}[1]{\footnote{#1}} % Use this for footnotes
\newcommand{\starttext}{
\setcounter{footnote}{0}
\renewcommand{\thefootnote}{\arabic{footnote}}}
\renewcommand{\theequation}{\thesection.\arabic{equation}}
\newcommand{\be}{\begin{equation}}
\newcommand{\bea}{\begin{eqnarray}}
\newcommand{\eea}{\end{eqnarray}}
\newcommand{\beq}{\begin{equation}}
\newcommand{\ee}{\end{equation}}
%
% The following macro is used to generate the header.
%
\newcommand{\lecture}[4]{
   \pagestyle{myheadings}
   \thispagestyle{plain}
   \newpage
   \setcounter{lecnum}{#1}
   \setcounter{page}{1}
   \noindent
   \begin{center}
   \framebox{
      \vbox{\vspace{2mm}
    \hbox to 6.28in { {\bf Thesis Notes - Aditya Vijaykumar
    \hfill Semester I 2017-18} }
       \vspace{4mm}
       \hbox to 6.28in { {\Large \hfill #1: #2  \hfill} }
       \vspace{2mm}
       
      }
   }
   \end{center}
   \markboth{Lecture #1: #2}{Lecture #1: #2}

   \vspace*{4mm}
}
%
% Convention for citations is authors' initials followed by the year.
% For example, to cite \bf a paper by Leighton and Maggs you would type
% \cite{LM89}, and to cite \bf a paper by Strassen you would type \cite{S69}.
% (To avoid bibliography problems, for now we redefine the \cite command.)
% Also commands that create \bf a suitable format for the reference list.
\renewcommand{\cite}[1]{[#1]}
\def\beginrefs{\begin{list}%
        {[\arabic{equation}]}{\usecounter{equation}
         \setlength{\leftmargin}{2.0truecm}\setlength{\labelsep}{0.4truecm}%
         \setlength{\labelwidth}{1.6truecm}}}
\def\endrefs{\end{list}}
\def\bibentry#1{\item[\hbox{[#1]}]}
\def\mean#1{\left< #1 \right>}

%Use this command for \bf a figure; it puts \bf a figure in wherever you want it.
%usage: \fig{NUMBER}{SPACE-IN-INCHES}{CAPTION}
\newcommand{\fig}[3]{
      \vspace{#2}
      \begin{center}
      Figure \thelecnum.#1:~#3
      \end{center}
  }
% Use these for theorems, lemmas, proofs, etc.
\newtheorem{theorem}{Theorem}[lecnum]
\newtheorem{lemma}[theorem]{Lemma}
\newtheorem{proposition}[theorem]{Proposition}
\newtheorem{claim}[theorem]{Claim}
\newtheorem{corollary}[theorem]{Corollary}
\newtheorem{definition}[theorem]{Definition}
%\newenvironment{proof}{{\bf Proof:}}{\hfill\rule{2mm}{2mm}}

% **** IF YOU WANT TO DEFINE ADDITIONAL MACROS FOR YOURSELF, PUT THEM HERE:
\newcommand{\be}{\begin{equation}}
\newcommand{\ee}{\end{equation}}
\newcommand\E{\mathbb{E}}


\begin{document}
%FILL IN THE RIGHT INFO.
%\lecture{**LECTURE-NUMBER**}{**TOPIC**}{**LECTURER**}{**LITE**}
\lecture{1}{Path Integrals and Miscellaneous Topics}{Aditya Vijaykumar}{scribe-name}
%\footnotetext{These notes are partially based on those of Nigel Mansell.}

% **** YOUR NOTES GO HERE:
\section{Preliminaries}
Here are some things that need to be kept in mind :-
$$X|x \ra = x|x \ra$$
$$\la x | x^\prime \ra = \delta(x-x^\prime)$$
$$\int dx |x \ra \la x| = 1$$
$$P|p \ra = p|p \ra$$
$$\la p | x \ra = \frac{1}{\sqrt{2 \pi \hbar}} \int dp \ e^{-ipx/\hbar}} = \la x | p \ra^*$$
$$f(x) = \la x | f \ra = \int dp \la x|p\ra \la p | f \ra = \frac{1}{\sqrt{2 \pi \hbar}} \int dp \ e^{-ipx/\hbar}} \la p | f \ra$$

\section{Ordering and Conventions}
In the Hmailtonian way of looking at things, we just promote the classical observables to operators while making the classical $\rightarrow$ quantum transition. But there is a slight problem - $[x_{cl},p_{cl}] = 0$, but $[x_{op},p_{op}] \ne 0$. Hence, we need to worry about the order in which we write the $p$'s and $x$'s.

There is no unique well-defined principle to do this. Two conventions are defined below :-

\begin{itemize}
 \item \textbf{Normal Ordering} - One just puts all $p$'s on the left of all the $x$'s. For example.
  $$ px^2 \xrightarrow{NO} px^2$$
  $$ xpx \xrightarrow{NO} px^2$$
  $$ x^2p \xrightarrow{NO} px^2$$
\item \textbf{Weyl Ordering} - One symmetrizes the product, and weights them equally.
  $$ px \xrightarrow{WO} \frac{px + xp}{2}$$
  $$ px^2 \xrightarrow{WO} \frac{px^2 + xpx + x^2p}{3}$$
  $$ x^m p^n \xrightarrow{WO} \frac{(\ldots)}{\binom{m+n}{m}}
\end{itemize}

How is the normal ordered Hamiltonian related to the classical Hamiltonian? We consider the matrix elements of $H^{NO}$, $\la x^\prime | H^{NO} | x \ra$.
$$\la x^\prime | H^{NO} | x \ra = \int \la x^\prime | p \ra \la p | H^{NO} | x \ra dp = \int dp \ e^{-\frac{ip(x-x^\prime)}{\hbar} H(p,x)$$





% Some general latex examples and examples making use of the
% macros follow.  
%**** IN GENERAL, BE BRIEF. LONG SCRIBE NOTES, NO MATTER HOW WELL WRITTEN,
%**** ARE NEVER READ BY ANYBODY.










% **** THIS ENDS THE EXAMPLES. DON'T DELETE THE FOLLOWING LINE:

\end{document}





