%
% This is the LaTeX template file for lecture notes for CS267,
% Applications of Parallel Computing.  When preparing 
% LaTeX notes for this class, please use this template.
%
% To familiarize yourself with this template, the body contains
% some examples of its use.  Look them over.  Then you can
% run LaTeX on this file.  After you have LaTeXed this file then
% you can look over the result either by printing it out with
% dvips or using xdvi.
%

\documentclass[twoside]{article}
\setlength{\oddsidemargin}{0.25 in}
\setlength{\evensidemargin}{-0.25 in}
\setlength{\topmargin}{-0.6 in}
\setlength{\textwidth}{6.5 in}
\setlength{\textheight}{8.5 in}
\setlength{\headsep}{0.75 in}
\setlength{\parindent}{0 in}
\setlength{\parskip}{0.1 in}

%
% ADD PACKAGES here:
%

\usepackage{amsmath,amsfonts,graphicx}
\usepackage{amsthm}
\usepackage{amstext}
\usepackage{amssymb}

\newcommand{\sgn}{{\rm sign}}
\newcommand{\ret}{\nonumber \\}
%
% The following commands set up the lecnum (lecture number)
% counter and make various numbering schemes work relative
% to the lecture number.
%
\newcounter{lecnum}
\renewcommand{\thepage}{\thelecnum-\arabic{page}}
\renewcommand{\thesection}{\thelecnum.\arabic{section}}
\renewcommand{\theequation}{\thelecnum.\arabic{equation}}
\renewcommand{\thefigure}{\thelecnum.\arabic{figure}}
\renewcommand{\thetable}{\thelecnum.\arabic{table}}

%
% The following macro is used to generate the header.
%
\newcommand{\lecture}[4]{
   \pagestyle{myheadings}
   \thispagestyle{plain}
   \newpage
   \setcounter{lecnum}{#1}
   \setcounter{page}{1}
   \noindent
   \begin{center}
   \framebox{
      \vbox{\vspace{2mm}
    \hbox to 6.28in { {\bf Thesis Notes - Aditya Vijaykumar
		\hfill Semester I 2017-18} }
       \vspace{4mm}
       \hbox to 6.28in { {\Large \hfill #1: #2  \hfill} }
       \vspace{2mm}
       
      }
   }
   \end{center}
   \markboth{Lecture #1: #2}{Lecture #1: #2}

   \vspace*{4mm}
}
%
% Convention for citations is authors' initials followed by the year.
% For example, to cite a paper by Leighton and Maggs you would type
% \cite{LM89}, and to cite a paper by Strassen you would type \cite{S69}.
% (To avoid bibliography problems, for now we redefine the \cite command.)
% Also commands that create a suitable format for the reference list.
\renewcommand{\cite}[1]{[#1]}
\def\beginrefs{\begin{list}%
        {[\arabic{equation}]}{\usecounter{equation}
         \setlength{\leftmargin}{2.0truecm}\setlength{\labelsep}{0.4truecm}%
         \setlength{\labelwidth}{1.6truecm}}}
\def\endrefs{\end{list}}
\def\bibentry#1{\item[\hbox{[#1]}]}

%Use this command for a figure; it puts a figure in wherever you want it.
%usage: \fig{NUMBER}{SPACE-IN-INCHES}{CAPTION}
\newcommand{\fig}[3]{
			\vspace{#2}
			\begin{center}
			Figure \thelecnum.#1:~#3
			\end{center}
	}
% Use these for theorems, lemmas, proofs, etc.
\newtheorem{theorem}{Theorem}[lecnum]
\newtheorem{lemma}[theorem]{Lemma}
\newtheorem{proposition}[theorem]{Proposition}
\newtheorem{claim}[theorem]{Claim}
\newtheorem{corollary}[theorem]{Corollary}
\newtheorem{definition}[theorem]{Definition}
%\newenvironment{proof}{{\bf Proof:}}{\hfill\rule{2mm}{2mm}}

% **** IF YOU WANT TO DEFINE ADDITIONAL MACROS FOR YOURSELF, PUT THEM HERE:
\newcommand{\be}{\begin{equation}}
\newcommand{\ee}{\end{equation}}
\newcommand\E{\mathbb{E}}

\begin{document}
%FILL IN THE RIGHT INFO.
%\lecture{**LECTURE-NUMBER**}{**TOPIC**}{**LECTURER**}{**LITE**}
\lecture{1}{Locality and the Lieb-Robinson Bounds}{Aditya Vijaykumar}{scribe-name}
%\footnotetext{These notes are partially based on those of Nigel Mansell.}

% **** YOUR NOTES GO HERE:

% Some general latex examples and examples making use of the
% macros follow.  
%**** IN GENERAL, BE BRIEF. LONG SCRIBE NOTES, NO MATTER HOW WELL WRITTEN,
%**** ARE NEVER READ BY ANYBODY.
Locality is a property of any system under consideration, and it manifests itself in the Hamiltonian of the system. For example, the Ising Model is local, in the sense that each particle (spin) interacts only with the spins adjacent to it in a one dimensional lattice. We could in principle also have theories (Hamiltonians) that exhibit locality upto a bound defined by an exponential decay or any other function.

It is hence apparent that to define locality, one needs to define the notion of a distance on the lattice. We could, most intuitively, define distance as ${\rm dist}(i,j)=|i-j|$, assuming open boundary conditions. We can in principle choose different metrics on the lattice, but our choice of metric should be such that the Hamiltonian remains local for the whole system. For example, in case of periodic boundary conditions the distance as above will not make the Hamiltonian local and we would need to define distance as ${\rm dist}(i,j)={\rm min}_n |i-j+nN|$.

By defining the distance metric on the lattice, we can also define the diameter of a set, and also the notion of distance betweem two sets.
\be
{\rm diam}(A)={\rm max}_{i,j \in A} {\rm dist}(i,j).
\ee
\be
{\rm dist}(A,B)={\rm min}_{i\in A,j\in B} {\rm dist}(i,j).
\ee
Diameter of a set gives the range of distance upto which a the Hamiltonian is effective, thereby describing locality of the system.

\textbf{NOTE:} The lattice need not be a regular one, and can be also thought of as the vertices of a graph.

\textbf{NOTE:} Heisenberg time evolution ie. $O(t)\equiv \exp(i H t) O \exp(-i H t)$ is considered everywhere.

\begin{theorem}
	Suppose for all sites $i$, the following holds:
	\begin{equation}
	\label{Hdecay}
	\sum_{X\ni i}
	\Vert H_X\Vert|X|\exp[{\mu\>{\rm diam}(X)}]\le s
	\end{equation}
	for some finite positive constants $\mu$ and $s$. Let $A_X,B_Y$ be operators supported on sets
	$X,Y$, respectively.  
	Then, if ${\rm dist}(X,Y)>0$,
	\begin{eqnarray}
	\label{lrbound}
	\Vert [A_X(t),B_Y]\Vert&\leq&
	2\Vert A_X\Vert \Vert B_Y\Vert \sum_{i \in X} \exp[-\mu\>{\rm dist}(i,Y)]\left[e^{2s|t|}-1\right]
	\\ \nonumber
	&\leq &
	2\Vert A_X\Vert \Vert B_Y\Vert |X| \exp[-\mu\>{\rm dist}(X,Y)]\left[e^{2s|t|}-1\right].
	\end{eqnarray}
\end{theorem}

Note that the summation in the assumption means the sum over all sets $X$ which contain a particular site $i$ satisfies the assumption (\ref{Hdecay}).

Suppose $A_X$ is some operator.  Let $B_l(X)$ denote the ball of radius $l$ about
set $X$.  That is, $B_l(X)$ is the set of sites $i$, such that ${\rm dist}(i,X)\leq l$.
If we define
\be
\label{localized}
A_X^l(t)=\int {\rm d}U U A_X(t) U^\dagger,
\ee
where the integral is over unitaries supported on the set of sites $\Lambda \setminus  B_l(X)$ with the Haar measure.
Then, $A_X^l$ is supported on $B_l(X)$.  Since $U A_X(t)  U^\dagger=A_X(t)+U[A_X(t),U^\dagger]$, we have
\be
\Vert A_X^l(t)-A_X(t) \leq \int {\rm d}U \Vert [A_X ,U] \Vert.
\ee
Using the Lieb-Robinson bound (\ref{lrbound}) to bound the right-hand side of the above equation,
we see that $A_X^l(t)$ is exponentially close to $A_X$ if $l$ is sufficiently large compared to $2st/\mu$.  Thus,
to exponential accuracy, we can approximate a time-evolved operator such as $A_X(t)$ by an operator supported on the set
$B_l(X)$.  That is, the ``leakage" of the operator outside the light-cone is small.

\section{Proof of the Lieb-Robinson Bound}
Let $A$ be supported on $X$ and $B$ be supported on $Y$. Let $\epsilon = t/N$ where $N$ is large, and lets define $t_n = nt/N = n\epsilon$. Then we have
\be
\left\Vert[A(t),B]\right\Vert-\left\Vert[A(0),B]\right\Vert 
=\sum_{i=0}^{N-1}\epsilon\times
\frac{\left\Vert[A(t_{n+1}),B]\right\Vert-\left\Vert[A(t_n),B]\right\Vert}{\epsilon}.
\label{sumid} 
\ee
We proceed to evaluate the summand on the right hand side.
\begin{eqnarray}
\left\Vert[A(t_{n+1}),B]\right\Vert-\left\Vert[A(t_n),B]\right\Vert
&=&\left\Vert[A(\epsilon),B(-t_n)]\right\Vert-\left\Vert[A,B(-t_n)]\right\Vert\ret
&=&\left\Vert[A+i\epsilon[H_\Lambda,A]+{\cal O}(\epsilon^2),B(-t_n)]\right\Vert
-\left\Vert[A,B(-t_n)]\right\Vert\ret
&\le&\left\Vert[A+i\epsilon[I_X,A],B(-t_n)]\right\Vert
-\left\Vert[A,B(-t_n)]\right\Vert+{\cal O}(\epsilon^2)
\label{difnorm}
\end{eqnarray}
In the last step, we have used the triangle inequality, and have defined $I_X=\sum_{Z:Z\cap X\ne \emptyset}H_Z$

Using the fact that $A+i\epsilon[I_X,A]=e^{i\epsilon I_X}Ae^{-i\epsilon I_X}+{\cal O}(\epsilon^2)$, we could further use triangle inequalities and write
\begin{eqnarray}
\left\Vert[A+i\epsilon[I_X,A],B(-t_n)]\right\Vert
&\le&\left\Vert[e^{i\epsilon I_X}Ae^{-i\epsilon I_X},B(-t_n)]\right\Vert
+{\cal O}(\epsilon^2)\ret
&\le&\left\Vert[A,e^{-i\epsilon I_X}B(-t_n)e^{i\epsilon I_X}]\right\Vert
+{\cal O}(\epsilon^2)\ret
&\le&\left\Vert[A,B(-t_n)-i\epsilon[I_X,B(-t_n)]]\right\Vert
+{\cal O}(\epsilon^2)\ret
&\le&\left\Vert[A,B(-t_n)]\right\Vert+\epsilon\left\Vert[A,[I_X,B(-t_n)]]\right\Vert
+{\cal O}(\epsilon^2). 
\end{eqnarray}
Therefore (\ref{difnorm}) becomes,
\be
\left\Vert[A(t_{n+1}),B]\right\Vert-\left\Vert[A(t_n),B]\right\Vert \le \epsilon\left\Vert[A,[I_X,B(-t_n)]]\right\Vert+{\cal O}(\epsilon^2)
\ee

Consider some $M,N$
\begin{eqnarray}
\left\Vert[M,N]\right\Vert &=& \left\Vert MN - NM\right\Vert\ret
&\le& 2\left\Vert MN \right\Vert\ret
&\le& 2\left\Vert M \right\Vert\left\Vert N \right\Vert
\end{eqnarray}
Using this, we write
\be
\left\Vert[A(t_{n+1}),B]\right\Vert-\left\Vert[A(t_n),B]\right\Vert \le 2\epsilon\left\Vert A \right\Vert \left\Vert [I_X(t_n),B] \right\Vert+{\cal O}(\epsilon^2)
\ee


\end{document}